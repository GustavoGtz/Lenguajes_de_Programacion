\documentclass{article}

\usepackage[spanish]{babel}

\usepackage[letterpaper,top=2cm,bottom=2cm,left=3cm,right=3cm,marginparwidth=1.75cm]{geometry}

\title{Lenguajes de Programación}
\author{Gutierrez Navarro Gustavo}

\begin{document}
\maketitle

\section{Tercer Problema:}
El último valor de retorno corresponde al número 0, esto quiere decir que el proceso de ejecución ha finalizado, 
por lo tanto, el orden de procesamiento se ejecutara a partir de aquí hacia arriba en orden según el Stack.
	
El primer bloque del Stack a ejecutar corresponde a la operación (- [] (- x 2)). El valor de "x" se encuentra en 
el bloque @1013, en este bloque el valor de "x" corresponde al valor 4, por otro lado, el valor que se encuentra 
dentro del cuadro se define según el último valor devuelto, en este caso el 0. En conclusión expresión termina 
como: (- 0 (- 4 2)). Al evaluar la expresión da como resultado -2, terminando así la ejecución del primer bloque 
en el Stack.
	
Pasando al segundo bloque del Stack, nos encontramos con la expresión (+ [] (+ y 4)), como acabamos de ver, el valor 
dentro del cuadro corresponde al último valor devuelto, es decir -2, por otro lado, la variable "y" se encuentra definida 
en el bloque @1668, dentro de este bloque la variable tiene asignado el valor de 6. Finalmente, la operación queda como: 
(+ (-2) (+ 6 4)), evaluando esta operación nos da el resultado de 8.
	
Para ya casi terminar, dentro del tercer bloque del Stack se evaluará la expresión (+ [] x). Contiene únicamente dos 
valores, el primero de ellos corresponde al último valor devuelto, en este caso corresponde a 8, el segundo elemento 
corresponde a la variable "x"; sin embargo, esta definición de variable no es la misma que en el bloque anterior, mientras 
que en el caso anterior estaba definida en el ambiente @1668, en esta llamada se encuentra definida en el ambiente @1467 
con el valor de "x" correspondiente a 3, por lo tanto, la expresión queda como (+ 8 3) que da como resultado el número 11.
	
Finalmente, el último bloque en orden de ejecución, simplemente nos pide el último valor devuelto, es decir, el número 11.
\section{Cuarto Problema:}
El algoritmo termina en el retorno con el valor 0. Posteriormente, sigue la ejecución del primer 
bloque del Stack según la prioridad de la estructura de datos. Dentro de este bloque se encuentra 
la expresión (+ [] x), los elementos que conforman a la operación son dos, en primer lugar es el 
último valor devuelto, es decir 0, mientras que el segundo valor corresponde a la variable "x", 
dicho valor se encuentra en el ambiente @1838, en este ambiente el valor de "x" está asignado al 
entero 2, por lo tanto, la operación que se evaluara es la siguiente: (+ 0 2), el resultado de
esto es 2.
	
Pasando al siguiente bloque de ejecución tenemos a la expresión (+ 4 [] y) una suma de tres 
elementos. El primero de ellos es simplemente un número ya definido, en cambio, sus otros dos 
elementos no lo están, el cuadro vacío hace referencia al último valor devuelto, en este caso 2 
y el valor de la variable "y" lo podemos encontrar en el ambiente @1092 que corresponde al 
número 5, en conclusión la operación resultaría en: (+ 4 2 5) con resultado 11.

El penúltimo bloque de ejecución tiene la expresión (+ [] x), sencillamente el primer elemento es 
el valor 11 que acabamos de obtener, mientras que la variable "x" está ligada al entorno @1964 en 
donde la "x" vale 3. Sencillamente, evaluamos la operación: (+ 11 3) cuyo resultado es 14.
	
Para finalizar este programa, únicamente se recoge el valor devuelto por el bloque superior, a lo 
que tenemos que este programa termina con el resultado del número 14.
\end{document}